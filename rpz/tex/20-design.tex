\section{Конструкторская часть}

В данном разделе будут сформулированы требования и ограничения к разрабатываемому методу, а также требования и ограничения к программному обеспечению. 
Разработан метод динамического отображения изменений пользовательского интерфейса на основе обработки изменений XML--файла.
Описаны основные этапы разработки в виде детализированной диаграммы IDEF0 и схем алгоритмов, а также изложены особенности излагаемого метода. 

\subsection{Требования и ограничения к разрабатываемому методу}

К методу динамического отображения изменений пользовательского интерфейса на основе обработки изменений XML--файла предъявляются следующие требования.
\begin{enumerate}[label*=\arabic*.]
	\item Обработка события, инициирующего изменения интерфейса во время выполнения.
	\item Преобразование XML--файла, организованного определенным образом, в UI--элементы.
	\item Размещение UI--элементов на экране.
\end{enumerate}
	
Представлено ограничение для разрабатываемого метода: при неправильной организации XML--файла не гарантируется корректное отображение элементов на экране.

\subsection{Требования к разрабатываемому программному обеспечению}

К разрабатываемому программному обеспечению предъявляются следующие требования.
\begin{enumerate}[label*=\arabic*.]
	\item Возможность обработки события, инициирующего изменение интерфейса.
	\item Возможность во время выполнения программы изменять интерфейс посредством внесения изменения в XML--файл по срабатыванию обработчика события.
	\item Возможность создавать интерфейсы с вложенностью элементов.
	\item Возможность создавать интерфейсы с базовыми UI--элементами.
	\item Возможность оперировать базовыми параметрами UI--элементов.
	\item Возможность комбинировать существующие методы создания интерфейса с разрабатываемым.
\end{enumerate}

% \newpage
\subsection{Основные этапы разрабатываемого метода}
На рисунке \ref{fig:a1} представлена диаграмма IDEF0 уровня А1 для разрабатываемого метода.

\begin{figure}[!htb]
	\centering
	\includegraphics[scale=0.4]{img/A1.pdf}
	\caption{IDEF0--диаграмма уровня А1}
	\label{fig:a1}
\end{figure}

\subsection{Схемы алгоритмов}

На \textbf{первом} этапе необходимо обработать событие, которое станет инициатором изменения интерфейса.
Данным событием может стать нажатие определенного сочетания клавиш.
Каждый проектируемый экран приложения должен быть связан с конкретным XML--файлом и заранее внесен в список наблюдаемых объектов. 
По свершении события происходит вызов функции обработки XML--файла для каждого такого экрана.


На \textbf{втором} этапе происходит обработка XML--файла. 
Для этого файл должен быть организован определенным образом, чтобы в дальнейшем функция обработки корректно распознавала UI--элементы, являющиеся базовыми для платформы: первый тег XML--файла содержит в себе название базового элемента и все его необходимые свойства и их значения (название свойства и значение разделяет знак равенства <<=>>, значение заключается в кавычки <<``''>>), второй тег содержит название базового элемента.
Для создания вложенности элементов среди свойств элемента--родителя необходимо указать теги дочернего элемента.
Разметка для дочернего элемента в данном случае задается по правилам, аналогичным правилам создания родительского элемента.

Например, чтобы в нативной разработке iOS получить UILabel, занимающий 100\% площади экрана и содержащий слово <<Hello>>, расположенное по левому краю, необходимо добавить в XML--файл разметку, представленную в листинге \ref{code:uilabel.xml}.

В рамках данного этапа происходит получение содержимого XML--файла, разделение файла на строки для получения списка его компонентов.
Далее происходит создание корневого представления, на котором будут размещены все элементы, полученные в ходе обработки компонентов XML--файла.
После чего корневое представление и компоненты передаются в функцию преобразование элементов XML--разметки в UI--элементы.
Для корректной работы функции необходимо иметь список всех базовых UI--элементов, с которыми предполагается взаимодействие метода, а также список свойств и атрибутов для каждого элемента.
Алгоритмы работы функций представлены на рисунках~~\ref{fig:createLayout} ---~~\ref{fig:createView}.  

\begin{code}
	\captionof{listing}{Разметка XML--файла для UILabel}
	\label{code:uilabel.xml}
	\inputminted
	[
	frame=single,
	framerule=0.5pt,
	framesep=10pt,
	fontsize=\small,
	tabsize=4,
	linenos,
	numbersep=5pt,
	xleftmargin=10pt,
	]
	{text}
	{code/uilabel.xml}
\end{code}
\clearpage
% \newpage
На рисунке \ref{fig:createLayout} представлен алгоритм функции создания представления, содержащего описанные в XML--файле элементы.
\begin{figure}[!htb]
	\centering
	\includegraphics[scale=0.92]{img/createLayout.pdf}
	\caption{Алгоритм функции создания представления, содержащего описанные в XML--файле элементы}
	\label{fig:createLayout}
\end{figure}
\clearpage
На рисунке \ref{fig:createView} представлен алгоритм функции преобразования в компоненты UI XML--разметки.
\begin{figure}[!htb]
	\centering
	\includegraphics[scale=0.55]{img/createView.pdf}
	\caption{Алгоритм функции преобразования XML--разметки в UI--компоненты}
	\label{fig:createView}
\end{figure}
\clearpage

Для \textbf{второго} этапа входным параметром XML--файл, организованный специальным образом, выходным~~---~~UI-элемент, включающий все представления, описанные в XML--файле.

На \textbf{третьем} этапе происходит размещение полученного UI--элемента на экране. 
Для корректного отображения обновлений данный этап должен быть включен в жизненный цикл корневого представления экрана.


\subsection*{Вывод}

В данном разделе были сформулированы требования и ограничения к разрабатываемому методу. 
Разработан метод динамического отображения изменений пользовательского интерфейса на основе обработки изменений XML--файла.
Описаны основные этапы разработки в виде детализированной диаграммы IDEF0 и схем алгоритмов, а также изложены особенности разработанного метода. 

\pagebreak