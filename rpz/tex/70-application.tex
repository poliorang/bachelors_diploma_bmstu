\section*{ПРИЛОЖЕНИЕ A}
\addcontentsline{toc}{section}{ПРИЛОЖЕНИЕ A}

\begin{code}
	\captionof{listing}{Класс, обеспечивающий перезагрузку во время выполнения}
	\label{code:ReloadManager.swift}
	\inputminted
	[
	frame=single,
	framerule=0.5pt,
	framesep=10pt,
	fontsize=\small,
	tabsize=4,
	linenos,
	numbersep=5pt,
	xleftmargin=10pt,
	]
	{text}
	{code/ReloadManager.swift}
\end{code}

\begin{code}
	\captionof{listing}{Класс, обеспечивающий перезагрузку во время выполнения --- продолжение}
	\label{code:ReloadManager2.swift}
	\inputminted
	[
	frame=single,
	framerule=0.5pt,
	framesep=10pt,
	fontsize=\small,
	tabsize=4,
	linenos,
	numbersep=5pt,
	xleftmargin=10pt,
	]
	{text}
	{code/ReloadManager2.swift}
\end{code}

\begin{code}
	\captionof{listing}{Функция определения списка событий, по свершению которых происходит перезагрузка}
	\label{code:events.swift}
	\inputminted
	[
	frame=single,
	framerule=0.5pt,
	framesep=10pt,
	fontsize=\small,
	tabsize=4,
	linenos,
	numbersep=5pt,
	xleftmargin=10pt,
	]
	{text}
	{code/events.swift}
\end{code}

\begin{code}
	\captionof{listing}{Функция определения списка событий, по свершению которых происходит перезагрузка --- продолжение}
	\label{code:events2.swift}
	\inputminted
	[
	frame=single,
	framerule=0.5pt,
	framesep=10pt,
	fontsize=\small,
	tabsize=4,
	linenos,
	numbersep=5pt,
	xleftmargin=10pt,
	]
	{text}
	{code/events2.swift}
\end{code}

\begin{code}
	\captionof{listing}{Функция, которая определяет инициирующее событие для обновления интерфейса}
	\label{code:overrideMethod.swift}
	\inputminted
	[
	frame=single,
	framerule=0.5pt,
	framesep=10pt,
	fontsize=\small,
	tabsize=4,
	linenos,
	numbersep=5pt,
	xleftmargin=10pt,
	]
	{text}
	{code/overrideMethod.swift}
\end{code}
\clearpage

\begin{code}
	\captionof{listing}{Функция, которая определяет инициирующее событие для обновления интерфейса --- продолжение}
	\label{code:overrideMethod2.swift}
	\inputminted
	[
	frame=single,
	framerule=0.5pt,
	framesep=10pt,
	fontsize=\small,
	tabsize=4,
	linenos,
	numbersep=5pt,
	xleftmargin=10pt,
	]
	{text}
	{code/overrideMethod2.swift}
\end{code}

\begin{code}
	\captionof{listing}{Перечисления и структуры, определяющие список базовых UI--элементов, их свойства и атрибуты}
	\label{components.swift}
	\inputminted
	[
	frame=single,
	framerule=0.5pt,
	framesep=10pt,
	fontsize=\small,
	tabsize=4,
	linenos,
	numbersep=5pt,
	xleftmargin=10pt,
	]
	{text}
	{code/components.swift}
\end{code}
\clearpage
\begin{code}
	\captionof{listing}{Перечисления и структуры, определяющие список базовых UI--элементов, их свойства и атрибуты -- продолжение}
	\label{components2.swift}
	\inputminted
	[
	frame=single,
	framerule=0.5pt,
	framesep=10pt,
	fontsize=\small,
	tabsize=4,
	linenos,
	numbersep=5pt,
	xleftmargin=10pt,
	]
	{text}
	{code/components2.swift}
\end{code}
\clearpage
\begin{code}
	\captionof{listing}{Перечисления и структуры, определяющие список базовых UI--элементов, их свойства и атрибуты --- продолжение}
	\label{components3.swift}
	\inputminted
	[
	frame=single,
	framerule=0.5pt,
	framesep=10pt,
	fontsize=\small,
	tabsize=4,
	linenos,
	numbersep=5pt,
	xleftmargin=10pt,
	]
	{text}
	{code/components3.swift}
\end{code}

\begin{code}
	\captionof{listing}{Класс, включающий функции преобразования XML--разметки в UI--элементы}
	\label{class.swift}
	\inputminted
	[
	frame=single,
	framerule=0.5pt,
	framesep=10pt,
	fontsize=\small,
	tabsize=4,
	linenos,
	numbersep=5pt,
	xleftmargin=10pt,
	]
	{text}
	{code/class.swift}
\end{code}
\clearpage
\begin{code}
	\captionof{listing}{Класс, включающий функции преобразования XML--разметки в UI--элементы --- продолжение}
	\label{class2.swift}
	\inputminted
	[
	frame=single,
	framerule=0.5pt,
	framesep=10pt,
	fontsize=\small,
	tabsize=4,
	linenos,
	numbersep=5pt,
	xleftmargin=10pt,
	]
	{text}
	{code/class2.swift}
\end{code}

\begin{code}
	\captionof{listing}{Класс, включающий функции преобразования XML--разметки в UI--элементы --- продолжение}
	\label{class3.swift}
	\inputminted
	[
	frame=single,
	framerule=0.5pt,
	framesep=10pt,
	fontsize=\small,
	tabsize=4,
	linenos,
	numbersep=5pt,
	xleftmargin=10pt,
	]
	{text}
	{code/class3.swift}
\end{code}

\clearpage
% \begin{code}
% 	\captionof{listing}{Класс, включающий функции преобразования XML--разметки в UI--элементы --- продолжение}
% 	\label{class4.swift}
% 	\inputminted
% 	[
% 	frame=single,
% 	framerule=0.5pt,
% 	framesep=10pt,
% 	fontsize=\small,
% 	tabsize=4,
% 	linenos,
% 	numbersep=5pt,
% 	xleftmargin=10pt,
% 	]
% 	{text}
% 	{code/class4.swift}
% \end{code}

\begin{code}
	\captionof{listing}{Класс, включающий функции преобразования XML--разметки в UI--элементы --- продолжение}
	\label{class5.swift}
	\inputminted
	[
	frame=single,
	framerule=0.5pt,
	framesep=10pt,
	fontsize=\small,
	tabsize=4,
	linenos,
	numbersep=5pt,
	xleftmargin=10pt,
	]
	{text}
	{code/class5.swift}
\end{code}

\begin{code}
	\captionof{listing}{Класс, включающий функции преобразования XML--разметки в UI--элементы --- продолжение}
	\label{class6.swift}
	\inputminted
	[
	frame=single,
	framerule=0.5pt,
	framesep=10pt,
	fontsize=\small,
	tabsize=4,
	linenos,
	numbersep=5pt,
	xleftmargin=10pt,
	]
	{text}
	{code/class6.swift}
\end{code}

\begin{code}
	\captionof{listing}{Класс, включающий функции преобразования XML--разметки в UI--элементы --- продолжение}
	\label{class7.swift}
	\inputminted
	[
	frame=single,
	framerule=0.5pt,
	framesep=10pt,
	fontsize=\small,
	tabsize=4,
	linenos,
	numbersep=5pt,
	xleftmargin=10pt,
	]
	{text}
	{code/class7.swift}
\end{code}

\begin{code}
	\captionof{listing}{Класс, включающий функции преобразования XML--разметки в UI--элементы --- продолжение}
	\label{class8.swift}
	\inputminted
	[
	frame=single,
	framerule=0.5pt,
	framesep=10pt,
	fontsize=\small,
	tabsize=4,
	linenos,
	numbersep=5pt,
	xleftmargin=10pt,
	]
	{text}
	{code/class8.swift}
\end{code}

\begin{code}
	\captionof{listing}{Класс, включающий функции преобразования XML--разметки в UI--элементы --- продолжение}
	\label{class9.swift}
	\inputminted
	[
	frame=single,
	framerule=0.5pt,
	framesep=10pt,
	fontsize=\small,
	tabsize=4,
	linenos,
	numbersep=5pt,
	xleftmargin=10pt,
	]
	{text}
	{code/class9.swift}
\end{code}

\begin{code}
	\captionof{listing}{Класс, включающий функции преобразования XML--разметки в UI--элементы --- продолжение}
	\label{class10.swift}
	\inputminted
	[
	frame=single,
	framerule=0.5pt,
	framesep=10pt,
	fontsize=\small,
	tabsize=4,
	linenos,
	numbersep=5pt,
	xleftmargin=10pt,
	]
	{text}
	{code/class10.swift}
\end{code}

\begin{code}
	\captionof{listing}{Класс, включающий функции преобразования XML--разметки в UI--элементы --- продолжение}
	\label{class11.swift}
	\inputminted
	[
	frame=single,
	framerule=0.5pt,
	framesep=10pt,
	fontsize=\small,
	tabsize=4,
	linenos,
	numbersep=5pt,
	xleftmargin=10pt,
	]
	{text}
	{code/class11.swift}
\end{code}

\begin{code}
	\captionof{listing}{Класс, включающий функции преобразования XML--разметки в UI--элементы --- продолжение}
	\label{class12.swift}
	\inputminted
	[
	frame=single,
	framerule=0.5pt,
	framesep=10pt,
	fontsize=\small,
	tabsize=4,
	linenos,
	numbersep=5pt,
	xleftmargin=10pt,
	]
	{text}
	{code/class12.swift}
\end{code}

\begin{code}
	\captionof{listing}{Класс, включающий функции преобразования XML--разметки в UI--элементы --- продолжение}
	\label{class13.swift}
	\inputminted
	[
	frame=single,
	framerule=0.5pt,
	framesep=10pt,
	fontsize=\small,
	tabsize=4,
	linenos,
	numbersep=5pt,
	xleftmargin=10pt,
	]
	{text}
	{code/class13.swift}
\end{code}

\begin{code}
	\captionof{listing}{Класс, включающий функции преобразования XML--разметки в UI--элементы --- продолжение}
	\label{class14.swift}
	\inputminted
	[
	frame=single,
	framerule=0.5pt,
	framesep=10pt,
	fontsize=\small,
	tabsize=4,
	linenos,
	numbersep=5pt,
	xleftmargin=10pt,
	]
	{text}
	{code/class14.swift}
\end{code}
\pagebreak

\section*{ПРИЛОЖЕНИЕ Б}
\addcontentsline{toc}{section}{ПРИЛОЖЕНИЕ Б}

\begin{table}[!htb]
 \caption{Результаты snapshot--тестирования}
 \label{table:tests1}
 \begin{center}
 \begin{tabular}{|p{0.6cm}|p{10cm}|p{5cm}|}
  \hline
   \bfseries № & \bfseries Описание теста & \bfseries Результат \\ \hline
   1 & UIView черного цвета, занимает верхнюю половину экрана & Тест пройден успешно  \\ \hline
   2 & UIView красного цвета, занимает нижнюю половину экрана & Тест пройден успешно  \\ \hline
   3 & UIView синего цвета, занимает левую половину экрана  & Тест пройден успешно  \\ \hline      
   4 & UIView оранжевого цвета, занимает правую половину экрана  & Тест пройден успешно  \\ \hline
   5 & UIView фиолетового цвета, ширина 20, высота 20, левая граница --- левая граница экрана, верхняя граница --- верхняя граница экрана & Тест пройден успешно  \\ \hline
   6 & UIView зеленого цвета, ширина 20, высота 20, правая граница --- правая граница экрана, верхняя граница --- верхняя граница экрана & Тест пройден успешно  \\ \hline
   7 & UIView серого цвета, занимает верхнюю половину экрана + UIView розового цвета, занимает нижнюю половину экрана & Тест пройден успешно  \\ \hline
   8 & UIView серого цвета, занимает левую половину экрана + UIView розового цвета, занимает правую половину экрана & Тест пройден успешно  \\ \hline
  \end{tabular}
 \end{center}
\end{table}

\begin{table}[!htb]
 \label{table:tests2}
 \begin{center}
  \caption{Результаты snapshot--тестирования --- продолжение}
 \begin{tabular}{|p{0.6cm}|p{10cm}|p{5cm}|}
  \hline
   \bfseries № & \bfseries Описание теста & \bfseries Результат \\ \hline
    9 & UILabel, цвет фона --- красный, текст --- <<Привет>>, отступ 100 от верхней границы экрана, отступ 100 от правой границы экрана, ширина и высота 100 & Тест пройден успешно  \\ 
   10 & UILabel, цвет фона --- красный, текст --- <<Привет>>, выравнивание текста по правому краю, отступ 100 от верхней границы экрана, отступ 100 от правой границы экрана, ширина и высота 100 & Тест пройден успешно  \\ \hline
   11 & UILabel, цвет фона --- красный, текст --- <<Привет>>, выравнивание текста по левому краю, отступ 100 от верхней границы экрана, отступ 100 от правой границы экрана, ширина и высота 100 & Тест пройден успешно  \\ \hline
   13 & UIView оранжевого цвета, занимает правую половину экрана + дочерний UILabel, цвет фона --- красный, текст --- <<Привет>>, выравнивание текста по правому краю, отступ 100 от верхней границы UIView, отступ 100 от правой границы UIView, ширина и высота 100 & Тест пройден успешно  \\ \hline
   14 & UIView оранжевого цвета, занимает правую половину экрана + дочерний UILabel, цвет фона --- красный, текст --- <<Привет>>, выравнивание текста по правому краю, отступ -100 от нижней границы UIView, отступ 100 от левой границы UIView, ширина и высота 100 & Тест пройден успешно  \\ \hline
  \end{tabular}
 \end{center}
\end{table}

\begin{table}[!htb]
 \label{table:tests3}
 \begin{center}
  \caption{Результаты snapshot--тестирования --- продолжение}
 \begin{tabular}{|p{0.6cm}|p{10cm}|p{5cm}|}
  \hline
   \bfseries № & \bfseries Описание теста & \bfseries Результат \\ \hline
   15 & UIView оранжевого цвета, занимает правую половину экрана + дочерний UILabel, цвет фона --- красный, текст --- <<Привет>>, выравнивание текста по правому краю, отступ 100 от нижней границы UIView, отступ 100 от левой границы UIView, ширина и высота 100 & Тест пройден успешно  \\ \hline
   16 & UIView красного цвета, 75\% экрана + дочерний UIView зеленого цвета, 50\% экрана + дочерний UIView коричневого цвета, который покрывает 25\% экрана  & Тест пройден успешно  \\ \hline   
   17 & UIView красного цвета, 75\% экрана + дочерний UIView зеленого цвета, 50\% экрана + дочерний UIView коричневого цвета, 25\% + дочерний UIView синего цвета, который покрывает 12\% экрана  & Тест пройден успешно  \\ \hline
   18 & UIView красного цвета, 75\% экрана + дочерний UIView зеленого цвета, 50\% экрана + дочерний UIView коричневого цвета, 25\% + дочерний UIView синего цвета, 12\% экрана + дочерний UIView оранжевого цвета, 6\% экрана  & Тест пройден успешно  \\ \hline
   19 & UIView красного цвета, 75\% экрана + дочерний UIView зеленого цвета, отступ 100 от правой границы родителя, ширина и высота 100 и второй дочерний UIView коричневого цвета, отступ -100 от левой границы родителя, ширина и высота 100 & Тест пройден успешно  \\ \hline
     \end{tabular}
 \end{center}
\end{table}

\begin{table}[!htb]
 \label{table:tests4}
 \begin{center}
  \caption{Результаты snapshot--тестирования --- продолжение}
 \begin{tabular}{|p{0.6cm}|p{10cm}|p{5cm}|}
  \hline
   \bfseries № & \bfseries Описание теста & \bfseries Результат \\ \hline
   20 & UIView красного цвета, 75\% экрана + дочерний UIView зеленого цвета, отступ 100 от правой границы родителя, ширина и высота 50, второй дочерний UIView коричневого цвета, отступ -100 от левой границы родителя, ширина и высота 50, третий дочерний UIView зеленого цвета, отступ 10 от верхний границы родителя, ширина и высота 50 & Тест пройден успешно  \\ \hline
   21 & UIView красного цвета, 75\% экрана + дочерний UIView зеленого цвета, отступ 100 от правой границы родителя, ширина и высота 50, второй дочерний UIView коричневого цвета, отступ -100 от левой границы родителя, ширина и высота 50, третий дочерний UIView зеленого цвета, отступ 10 от верхний границы родителя, ширина и высота 50 + дочерний UILabel, цвет фона --- красный, текст --- <<Привет>>, выравнивание текста по правому краю, отступ 10 от верхней границы родителя & Тест пройден успешно  \\ \hline 
   22 & UIView красного цвета, 75\% экрана + дочерний UIView зеленого цвета, отступ 100 от правой границы родителя, ширина и высота 50, второй дочерний UIView коричневого цвета, отступ -100 от левой границы родителя, ширина и высота 50, третий дочерний UIView зеленого цвета & Тест пройден успешно  \\ \hline \end{tabular}
 \end{center}
\end{table}

\pagebreak

