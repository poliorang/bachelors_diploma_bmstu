\section*{ЗАКЛЮЧЕНИЕ}
\addcontentsline{toc}{section}{ЗАКЛЮЧЕНИЕ}

В ходе выполнения дипломной работы был разработан метода динамического отображения изменений пользовательского интерфейса на основе обработки изменений XML–файла.

Для достижения поставлен­ной цели были решены следующие задачи:

\begin{itemize}[label=---]
	\item выявлены критерии, по которым можно классифицировать методы отображения пользовательского интерфейса;
	\item рассмотрены и классифицированы согласно выдвинутым критериям существующие методы отображения интерфейса;
	\item разработан метод динамического отображения изменений интерфейса на основе обработки изменений XML--файла;
	\item разработано программное обеспечение, реализующее представленный метод;
	\item произведено сравнение скорости внесения изменений в интерфейс для разработанной и существующей реализаций.
\end{itemize}

В силу отсутствия функции горячей перезагрузки в нативной мобильной разработке, программное обеспечение, реализующее представленный метод, было разработано для создания интерфейсов на платформе iOS.
ПО предоставляет возможности для создания интерфейсов с базовыми UI--элементами, а также внесения изменений в UI во время выполнения программы.

В результате исследования было установлено, что разработанный метод позволяет колоссально сократить время разработки за счет возможности внесения изменения в интерфейс во время выполнения приложения.
Время, затрачиваемое на проектирование интерфейса с применением разработанного метода, в полторы тысячи раз меньше, нежели время, затрачиваемое при использовании существующего метода.

\pagebreak