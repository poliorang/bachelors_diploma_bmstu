\section*{ЗАКЛЮЧЕНИЕ}
\addcontentsline{toc}{section}{ЗАКЛЮЧЕНИЕ}

В ходе выполнения дипломной работы был исследован метод сжатия страниц оперативной памяти в ядре Linux и была разработана его оптимизации.

Для достижения поставлен­ной цели были решены следующие задачи:

\begin{itemize}
	\item рассмотрены методы увеличения количества оперативной памяти;
	\item дана характеристика архитектурам ядер современных операционных систем;
	\item изучены подходы, структуры данных и API \cite{api} в ядре Linux, позволяющие управлять подсистемой памяти;
	\item описана работа модуля сжатия оперативной памяти в ядре Linux;
	\item разработана оптимизацию для данного модуля;
	\item спроектирована структура программного обеспечения, реализующего оптимизацию модуля сжатия оперативной памяти;
	\item сравнен метод сжатия страниц оперативной памяти с оптимизацией и без.
\end{itemize}

Разработанная оптимизация была отправлена в качестве RFC письма разработчикам модуля ядра zram. В качестве дальнейшего развития было предложено вычисление порогового значения информационной энтропии не статически (на стадии компиляции), а динамически (во время работы блочного устройства).

\pagebreak