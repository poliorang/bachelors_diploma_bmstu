\section*{РЕФЕРАТ}
\addcontentsline{toc}{section}{РЕФЕРАТ}

Расчетно--пояснительная записка \pageref{LastPage} с., \totalfigures\ рис., \totaltables\ табл., 59 ист.

\textbf{Ключевые слова:} пользовательский интерфейс, горячая перезагрузка, XML.

Объектом исследования данной работы является динамическое отображение изменений пользовательского интерфейса.
Существует множество методов отображения интерфейса, однако не каждый из них предоставляет возможность мгновенно отображать изменения при внесении коррективов в код. 
Целью данной работы является разработка метода динамического отображения изменений пользовательского интерфейса на основе обработки изменений XML--файла.

Для достижения поставленной цели были решены следующие задачи:
\begin{itemize}[label=---]
	\item выявлены критерии, по которым можно классифицировать методы отображения пользовательского интерфейса;
	\item рассмотрены и классифицированы согласно выдвинутым критериям существующие методы отображения интерфейса;
	\item разработан метод динамического отображения изменений интерфейса на основе обработки изменений XML--файла;
	\item разработано программное обеспечение, реализующее представленный метод;
	\item проведено сравнение скорости внесения изменений в интерфейс разработанной и существующей реализаций.
\end{itemize}

Поставленная цель достигнута: в ходе дипломной работы был разработан метод динамического отображения изменений пользовательского интерфейса на основе обработки изменений XML--файла.
Разработанный метод в сравнении с аналогами повышает скорость внесений изменений в интерфейс: это достигается благодаря реализации функции горячей перезагрузки.


\pagebreak