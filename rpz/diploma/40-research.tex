\section{Исследовательская часть}

В данном разделе будет проведено исследование эффективности реализованного метода путем сравнения скорости внесения изменений в интерфейс с существующими реализациями.

\subsection{Технические характеристики}

Технические характеристики машины, на которой производились исследования:
\begin{itemize}[label=---]
	\item операционная система: macOS Sonoma 14.2.1;
	\item оперативная память: 16 Гб;
	\item процессор: Apple M1 Pro;
	 \item количество ядер: 8.
\end{itemize}

\subsection{Постановка исследования}

В рамках исследования была оценена эффективность реализованного метода путем сравнения скорости внесения изменений в интерфейс с существующими реализациями.
В качестве уже существующего метода отображения пользовательского интерфейса в нативной разработке iOS выбран фреймворк UIKit, не предоставляющий возможности горячей перезагрузки.

Для исследования работоспособности созданного программного обеспечения и оценки времени применения изменений UI были выбраны три варианта интерфейса:
\begin{itemize}[label=---]
	\item простой интерфейс, содержащий одно представление UIView;
	\item интерфейс средней сложности, содержащий три представления UIView, находящихся на одном уровне в иерархии, одно из которых содержит дочерний UILabel;
	\item сложный интерфейс, содержащий три представления UIView, каждый из которых содержит в себе два дочерних представления, внутри которых находится UIView, включающий в себя UILabel.
\end{itemize}	

Данные варианты интерфейса спроектированы с применением UIKit и с использованием разработанного программного обеспечения.

Также были классифицированы типы изменений, вносимых в интерфейс:
\begin{itemize}[label=---]
	\item простое: изменения одного атрибута или свойства для одного представления интерфейса;
	\item средней сложности: изменения двух --- пяти атрибутов или свойств для двух --- пяти представления интерфейса и/или добавление или удаление одного дочернего представления;
	\item сложное: изменения шести --- десяти атрибутов или свойств для шести --- десяти представлениий интерфейса и/или добавление или удаление двух или трех дочерних представлений.
\end{itemize}		

Исследование предполагает для каждого из вариантов интерфейса построение первоначальной версии с использованием UIKit, и с использованием разработанного программного обеспечения.
Далее происходило внесение одного из видов изменений в интерфейс.
После чего засекалось время, за которое интерфейс перейдет из первоначального состояния в измененное. 

\subsection{Результаты исследования}

\subsection*{Вывод}

В результате исследования было установлено что коэффициент сжатия сильно зависит от входных данных. Так, например, файл формата pdf практически не сжался как с включенной энтропийной оптимизацией, так и без.

Было подобран пороговое значения энтропии для алгоритма zstd -- 100 000. Таким образом, страницы, энтропия которых более 100 000 будут храниться в блочном устройстве несжатые. 

Оптимизация метода сжатия ускоряет процесс преобразования данных от 1.5 до 4 раз. Чем меньше процент сжатия исходных данных без включенной оптимизации, тем больше ускоряется процесс сжатия с включенной оптимизацией. Файл формата pdf потерял маленькую долю сжатия (коэффициент сжатия стал 1.002, вместо 1.004), но при этом, процесс преобразования данных ускорился в 4 раза.

Разносортные данные упакованные в единый файл сжимаются в среднем на 25\%. Сжатие с энтропийной оптимизацией происходит в 1.9 раза быстрее, чем без. При этом, потеря в сжатии составляет менее процента.

При сжатии директории /lib, в которой хранятся различные системные библиотеки, 25\%-ый выигрыш по времени достигается с помощью потери 5.5\% сжатого объема (итогового размера).

Подводя итог, можно сделать вывод, что разработанная оптимизация программного обеспечения ускоряет процесс сжатия в несколько раз (1.5-4 раза), при этом потеря в сжатии не крайне малы: от 1\% до 16\%, в зависимости от входных данных. 

Полученные результаты, вместе с модификацией модуля zram, были отправлены в качестве RFC (англ. request for comments \cite{rfc}) письма мейнтейнерам модуля ядра zram \cite{rfc-kernel-patch}.

\pagebreak