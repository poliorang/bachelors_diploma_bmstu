\section*{ПРИЛОЖЕНИЕ A}
\addcontentsline{toc}{section}{ПРИЛОЖЕНИЕ A}

\begin{lstlisting}[caption={Класс, обеспечивающий перезагрузку во время выполнения}]

class ReloadManager {
    private static var observersWeak = [ObserverWeak]()

    private struct ObserverWeak {
        weak var observer: LayoutInTimeViewController?
    }

    static var observers: [LayoutInTimeViewController] {
        return observersWeak.compactMap { $0.observer }
    }

    static func addObserver(_ observer: LayoutInTimeViewController) {
        var alreadyRegistered = false
        observersWeak = observersWeak.filter {
            guard let o = $0.observer else { return false }
            if o === observer { alreadyRegistered = true }
            return true
        }
        if alreadyRegistered {
            return
        }
        observersWeak.append(ObserverWeak(observer: observer))

        #if arch(arm64) || arch(x86_64)

            if !UIResponder.handlerInstalled {
                overrideMethod(#selector(getter: UIResponder.keyCommands),
                               of: UIResponder.self,
                               with: #selector(UIResponder.layoutItTimeKeyCommands)
                )
                UIResponder.handlerInstalled = true
            }

        #endif
    }

    static func reload() {
        for observer in observers {
            observer.reload()
        }
    }
}
\end{lstlisting}

\begin{lstlisting}[caption={Функция определения списка событий, по свершению которых происходит перезагрузка}]
#if arch(arm64) || arch(x86_64)
    private extension UIResponder {
        static var handlerInstalled = false

        @objc func layoutItTimeKeyCommands() -> [UIKeyCommand]? {

            return (layoutItTimeKeyCommands() ?? []) + [
                UIKeyCommand(
                    input: "t",
                    modifierFlags: .command,
                    action: #selector(reloadInTime)
                ),
            ]
        }

        @objc private func reloadInTime() {
            ReloadManager.reload()
        }
    }
#endif
\end{lstlisting}

\begin{lstlisting}[caption={Класс, обеспечивающий перезагрузку по время выполнения}]
func overrideMethod(_ selectorA: Selector, of sourceClass: AnyClass, with selectorB: Selector) {
    let newMethod = class_getInstanceMethod(sourceClass, selectorB)!
    let oldMethod = class_getInstanceMethod(sourceClass, selectorA)!
    let inheritedImplementation = class_getInstanceMethod(class_getSuperclass(sourceClass), selectorA)
        .map(method_getImplementation)
    if method_getImplementation(oldMethod) == inheritedImplementation {
        let types = method_getTypeEncoding(oldMethod)
        class_addMethod(sourceClass, selectorA, method_getImplementation(newMethod), types)
        return
    }
    method_exchangeImplementations(oldMethod, newMethod)
}
\end{lstlisting}

\begin{lstlisting}[caption={Внесение UIViewController в список наблюдаемых объектов}]
ReloadManager.addObserver(self)
\end{lstlisting}

\begin{lstlisting}[caption={Перечисления и структуры, определяющие список базовых UI--элементов, их свойства и атрибуты}]
enum ComponentType {
    case uiView
    case uiLabel
    case uiButton

    var startTag: String {
        switch self {
        case .uiView:
            return "UIView"
        case .uiLabel:
            return "UILabel"
        case .uiButton:
            return "UIButton"
        }
    }

    var endTag: String {
        switch self {
        case .uiView:
            return "/UIView"
        case .uiLabel:
            return "/UILabel"
        case .uiButton:
            return "/UIButton"
        }
    }
}

struct Anchors {
    public var topAnchorConstant: CGFloat?
    public var bottomAnchorConstant: CGFloat?
    public var leftAnchorConstant: CGFloat?
    public var rightAnchorConstant: CGFloat?
    public var height: CGFloat?
    public var width: CGFloat?
}

enum ComponentPropertyType {
    case width
    case height
    case topAnchor
    case bottomAnchor
    case leftAnchor
    case rightAnchor
    case backgroundColor
    case textAlignment
    case text

    var value: String {
        switch self {
        case .width:
            "width"
        case .height:
            "height"
        case .topAnchor:
            "topAnchor"
        case .bottomAnchor:
            "bottomAnchor"
        case .leftAnchor:
            "leftAnchor"
        case .rightAnchor:
            "rightAnchor"
        case .backgroundColor:
            "backgroundColor"
        case .textAlignment:
            "textAlignment"
        case .text:
            "text"
        }
    }
}
\end{lstlisting}

\begin{lstlisting}[caption={Функция определения списка событий, по свершению которых происходит перезагрузка}]
class LayoutInTime {
    public func createLayout(rootView: UIView, from xml: String) {
        guard let clearContent = parseXMLElements(xmlContent: xml) else {
            return
        }
        let separatedClearContent = clearContent.components(separatedBy: "\n")
        let joinedClearContent = separatedClearContent.joined(separator: "=")
        let components = joinedClearContent.components(separatedBy: "=")

        print(components)
        _ = createView(rootView: rootView, from: components)
    }

    private func createView(rootView: UIView, from components: [String]) -> Int {
        var i = 0
        var finalTagIndex = 0
        while i < components.count {
            switch components[i] {
            case ComponentType.uiView.startTag:
                var j = i + 1
                var uiView = UIView().autolayout()
                var anchors = Anchors()
                while components[j] != ComponentType.uiView.endTag {
                    switch components[j] {
                    case ComponentType.uiView.startTag,
                        ComponentType.uiLabel.startTag,
                        ComponentType.uiButton.startTag:
                        j += createView(rootView: uiView, from: Array(components[j..<components.count]))

                    case ComponentPropertyType.width.value:
                        incrementIndex(index: &j)
                        anchors.width = setWidthHeight(property: components[j], isHeight: false)
                    case ComponentPropertyType.height.value:
                        incrementIndex(index: &j)
                        anchors.height = setWidthHeight(property: components[j], isHeight: true)

                    case ComponentPropertyType.topAnchor.value:
                        incrementIndex(index: &j)
                        anchors.topAnchorConstant = Double(components[j]) ?? 0.0
                    case ComponentPropertyType.bottomAnchor.value:
                        incrementIndex(index: &j)
                        anchors.bottomAnchorConstant = Double(components[j]) ?? 0.0
                    case ComponentPropertyType.leftAnchor.value:
                        incrementIndex(index: &j)
                        anchors.leftAnchorConstant = Double(components[j]) ?? 0.0
                    case ComponentPropertyType.rightAnchor.value:
                        incrementIndex(index: &j)
                        anchors.rightAnchorConstant = Double(components[j]) ?? 0.0

                    case ComponentPropertyType.backgroundColor.value:
                        incrementIndex(index: &j)
                        uiView = setBackgroundColor(component: uiView, property: components[j])

                    default:
                        continue
                    }
                    incrementIndex(index: &j)
                }
                finalTagIndex = j
                rootView.addSubview(uiView)
                setAnchors(rootView: rootView, childView: uiView, anchors: anchors)
                incrementIndexByIndex(indexI: &i, by: j)

            case ComponentType.uiLabel.startTag:
                var j = i + 1
                var uiLabel = UILabel().autolayout()
                var anchors = Anchors()
                while components[j] != ComponentType.uiLabel.endTag {
                    print(components[j])
                    switch components[j] {
                    case ComponentType.uiView.startTag,
                        ComponentType.uiLabel.startTag,
                        ComponentType.uiButton.startTag:
                        j += createView(rootView: uiLabel, from: Array(components[j..<components.count]))

                    case ComponentPropertyType.width.value:
                        incrementIndex(index: &j)
                        anchors.width = setWidthHeight(property: components[j], isHeight: false)
                    case ComponentPropertyType.height.value:
                        incrementIndex(index: &j)
                        anchors.height = setWidthHeight(property: components[j], isHeight: true)

                    case ComponentPropertyType.topAnchor.value:
                        incrementIndex(index: &j)
                        anchors.topAnchorConstant = Double(components[j]) ?? 0.0
                    case ComponentPropertyType.bottomAnchor.value:
                        incrementIndex(index: &j)
                        anchors.bottomAnchorConstant = Double(components[j]) ?? 0.0
                    case ComponentPropertyType.leftAnchor.value:
                        incrementIndex(index: &j)
                        anchors.leftAnchorConstant = Double(components[j]) ?? 0.0
                    case ComponentPropertyType.rightAnchor.value:
                        incrementIndex(index: &j)
                        anchors.rightAnchorConstant = Double(components[j]) ?? 0.0

                    case ComponentPropertyType.backgroundColor.value:
                        incrementIndex(index: &j)
                        uiLabel = setBackgroundColor(component: uiLabel, property: components[j]) as! UILabel

                    case ComponentPropertyType.textAlignment.value:
                        incrementIndex(index: &j)
                        setTextAligment(component: &uiLabel, property: components[j])
                    case ComponentPropertyType.text.value:
                        incrementIndex(index: &j)
                        uiLabel.text = components[j]

                    default:
                        continue
                    }
                    incrementIndex(index: &j)
                }
                finalTagIndex = j
                rootView.addSubview(uiLabel)
                setAnchors(rootView: rootView, childView: uiLabel, anchors: anchors)
                incrementIndexByIndex(indexI: &i, by: j)

            case ComponentType.uiButton.startTag:
                var j = i + 1
                var uiButton = UIButton().autolayout()
                var anchors = Anchors()
                while components[j] != ComponentType.uiButton.endTag {
                    switch components[j] {
                    case ComponentType.uiView.startTag,
                        ComponentType.uiLabel.startTag,
                        ComponentType.uiButton.startTag:
                        j += createView(rootView: uiButton, from: Array(components[j..<components.count]))

                    case ComponentPropertyType.width.value:
                        incrementIndex(index: &j)
                        anchors.width = setWidthHeight(property: components[j], isHeight: false)
                    case ComponentPropertyType.height.value:
                        incrementIndex(index: &j)
                        anchors.height = setWidthHeight(property: components[j], isHeight: true)

                    case ComponentPropertyType.topAnchor.value:
                        incrementIndex(index: &j)
                        anchors.topAnchorConstant = Double(components[j]) ?? 0.0
                    case ComponentPropertyType.bottomAnchor.value:
                        incrementIndex(index: &j)
                        anchors.bottomAnchorConstant = Double(components[j]) ?? 0.0
                    case ComponentPropertyType.leftAnchor.value:
                        incrementIndex(index: &j)
                        anchors.leftAnchorConstant = Double(components[j]) ?? 0.0
                    case ComponentPropertyType.rightAnchor.value:
                        incrementIndex(index: &j)
                        anchors.rightAnchorConstant = Double(components[j]) ?? 0.0

                    case ComponentPropertyType.backgroundColor.value:
                        incrementIndex(index: &j)
                        uiButton = setBackgroundColor(component: uiButton, property: components[j]) as! UIButton

                    default:
                        continue
                    }
                    incrementIndex(index: &j)
                }
                finalTagIndex = j
                rootView.addSubview(uiButton)
                setAnchors(rootView: rootView, childView: uiButton, anchors: anchors)
                incrementIndexByIndex(indexI: &i, by: j)
            default:
                incrementIndex(index: &i)
            }
        }
        return finalTagIndex
    }

    private func incrementIndex(index: inout Int) {
        index += 1
    }

    private func incrementIndexByIndex(indexI: inout Int, by indexJ: Int) {
        indexI = indexJ + 1
    }

    private func createBaseView() -> UIView {
        UIView(frame: CGRect(
            x: 0,
            y: 0,
            width: UIScreen.main.bounds.width,
            height: UIScreen.main.bounds.height)
        )
    }
}

extension LayoutInTime {
    private func setTextAligment(component: inout UILabel, property: String) {
        switch property {
        case "left":
            component.textAlignment = NSTextAlignment(rawValue: 0)!
        case "center":
            component.textAlignment = NSTextAlignment(rawValue: 1)!
        case "right":
            component.textAlignment = NSTextAlignment(rawValue: 2)!
        default:
            break
        }
    }

    private func setBackgroundColor(component: UIView, property: String) -> UIView {
        switch property {
        case "white":
            component.backgroundColor = .white
        case "black":
            component.backgroundColor = .black
        case "green":
            component.backgroundColor = .green
        case "red":
            component.backgroundColor = .red
        case "blue":
            component.backgroundColor = .blue
        case "brown":
            component.backgroundColor = .brown
        case "cyan":
            component.backgroundColor = .cyan
        case "darkGray":
            component.backgroundColor = .darkGray
        case "gray":
            component.backgroundColor = .gray
        case "systemPink":
            component.backgroundColor = .systemPink
        case "yellow":
            component.backgroundColor = .yellow
        case "purple":
            component.backgroundColor = .purple
        case "orange":
            component.backgroundColor = .orange
        default:
            break
        }

        return component
    }

    private func setWidthHeight(property: String, isHeight: Bool) -> CGFloat {
        if property.contains("%") {
            return CGFloat(Double(
                property.replacingOccurrences(of: "%", with: "")
            ) ?? 0) / 100.0 * (isHeight ? UIScreen.main.bounds.height : UIScreen.main.bounds.width)
        }
        return CGFloat(Double(property) ?? 0)
    }

    private func setAnchors(rootView: UIView, childView: UIView, anchors: Anchors) {
        if let topAnchorConstant = anchors.topAnchorConstant {
            NSLayoutConstraint.activate([
                childView.topAnchor.constraint(equalTo: rootView.topAnchor, constant: topAnchorConstant)
            ])
        }
        if let leftAnchorConstant = anchors.leftAnchorConstant {
            NSLayoutConstraint.activate([
                childView.leftAnchor.constraint(equalTo: rootView.leftAnchor, constant: leftAnchorConstant)
            ])
        }
        if let bottomAnchorConstant = anchors.bottomAnchorConstant {
            NSLayoutConstraint.activate([
                childView.bottomAnchor.constraint(equalTo: rootView.bottomAnchor, constant: bottomAnchorConstant)
            ])
        }
        if let rightAnchorConstant = anchors.rightAnchorConstant {
            NSLayoutConstraint.activate([
                childView.rightAnchor.constraint(equalTo: rootView.rightAnchor, constant: rightAnchorConstant)
            ])
        }
        if let height = anchors.height {
            NSLayoutConstraint.activate([
                childView.heightAnchor.constraint(equalToConstant: height)
            ])
        }
        if let width = anchors.width {
            NSLayoutConstraint.activate([
                childView.widthAnchor.constraint(equalToConstant: width)
            ])
        }
    }
}
\end{lstlisting}

\pagebreak

\section*{ПРИЛОЖЕНИЕ Б}
\addcontentsline{toc}{section}{ПРИЛОЖЕНИЕ Б}

\begin{table}[!htb]
 \label{table:tests1}
 \begin{center}
  \caption{Результаты snapshot--тестирования}
 \begin{tabular}{|p{0.6cm}|p{10cm}|p{5cm}|}
  \hline
   \bfseries № & \bfseries Описание теста & \bfseries Результат \\ \hline
   1 & UIView черного цвета, занимает верхнюю половину экрана & Тест пройден успешно  \\ \hline
   2 & UIView красного цвета, занимает нижнюю половину экрана & Тест пройден успешно  \\ \hline
   3 & UIView синего цвета, занимает левую половину экрана  & Тест пройден успешно  \\ \hline      
   4 & UIView оранжевого цвета, занимает правую половину экрана  & Тест пройден успешно  \\ \hline
   5 & UIView фиолетового цвета, ширина 20, высота 20, левая граница --- левая граница экрана, верхняя граница --- верхняя граница экрана & Тест пройден успешно  \\ \hline
   6 & UIView зеленого цвета, ширина 20, высота 20, правая граница --- правая граница экрана, верхняя граница --- верхняя граница экрана & Тест пройден успешно  \\ \hline
   7 & UIView серого цвета, занимает верхнюю половину экрана + UIView розового цвета, занимает нижнюю половину экрана & Тест пройден успешно  \\ \hline
   8 & UIView серого цвета, занимает левую половину экрана + UIView розового цвета, занимает правую половину экрана & Тест пройден успешно  \\ \hline
  \hline
  \end{tabular}
 \end{center}
\end{table}

\begin{table}[!htb]
 \label{table:tests2}
 \begin{center}
  \caption{Результаты snapshot--тестирования --- продолжение}
 \begin{tabular}{|p{0.6cm}|p{10cm}|p{5cm}|}
  \hline
   \bfseries № & \bfseries Описание теста & \bfseries Результат \\ \hline
    9 & UILabel, цвет фона --- красный, текст --- <<Привет>>, отступ 100 от верхней границы экрана, отступ 100 от правой границы экрана, ширина и высота 100 & Тест пройден успешно  \\ 
   10 & UILabel, цвет фона --- красный, текст --- <<Привет>>, выравнивание текста по правому краю, отступ 100 от верхней границы экрана, отступ 100 от правой границы экрана, ширина и высота 100 & Тест пройден успешно  \\ \hline
   11 & UILabel, цвет фона --- красный, текст --- <<Привет>>, выравнивание текста по левому краю, отступ 100 от верхней границы экрана, отступ 100 от правой границы экрана, ширина и высота 100 & Тест пройден успешно  \\ \hline
   13 & UIView оранжевого цвета, занимает правую половину экрана + дочерний UILabel, цвет фона --- красный, текст --- <<Привет>>, выравнивание текста по правому краю, отступ 100 от верхней границы UIView, отступ 100 от правой границы UIView, ширина и высота 100 & Тест пройден успешно  \\ \hline
   14 & UIView оранжевого цвета, занимает правую половину экрана + дочерний UILabel, цвет фона --- красный, текст --- <<Привет>>, выравнивание текста по правому краю, отступ -100 от нижней границы UIView, отступ 100 от левой границы UIView, ширина и высота 100 & Тест пройден успешно  \\ \hline
  \end{tabular}
 \end{center}
\end{table}

\begin{table}[!htb]
 \label{table:tests3}
 \begin{center}
  \caption{Результаты snapshot--тестирования --- продолжение}
 \begin{tabular}{|p{0.6cm}|p{10cm}|p{5cm}|}
  \hline
   \bfseries № & \bfseries Описание теста & \bfseries Результат \\ \hline
   15 & UIView оранжевого цвета, занимает правую половину экрана + дочерний UILabel, цвет фона --- красный, текст --- <<Привет>>, выравнивание текста по правому краю, отступ 100 от нижней границы UIView, отступ 100 от левой границы UIView, ширина и высота 100 & Тест пройден успешно  \\ \hline
   16 & UIView красного цвета, 75\% экрана + дочерний UIView зеленого цвета, 50\% экрана + дочерний UIView коричневого цвета, 25\% экрана  & Тест пройден успешно  \\ \hline   
   17 & UIView красного цвета, 75\% экрана + дочерний UIView зеленого цвета, 50\% экрана + дочерний UIView коричневого цвета, 25\% + дочерний UIView синего цвета, 12\% экрана  & Тест пройден успешно  \\ \hline
   18 & UIView красного цвета, 75\% экрана + дочерний UIView зеленого цвета, 50\% экрана + дочерний UIView коричневого цвета, 25\% + дочерний UIView синего цвета, 12\% экрана + дочерний UIView оранжевого цвета, 6\% экрана  & Тест пройден успешно  \\ \hline
   19 & UIView красного цвета, 75\% экрана + дочерний UIView зеленого цвета, отступ 100 от правой границы родителя, ширина и высота 100 и второй дочерний UIView коричневого цвета, отступ -100 от левой границы родителя, ширина и высота 100 & Тест пройден успешно  \\ \hline
     \end{tabular}
 \end{center}
\end{table}

\begin{table}[!htb]
 \label{table:tests4}
 \begin{center}
  \caption{Результаты snapshot--тестирования --- продолжение}
 \begin{tabular}{|p{0.6cm}|p{10cm}|p{5cm}|}
  \hline
   \bfseries № & \bfseries Описание теста & \bfseries Результат \\ \hline
   20 & UIView красного цвета, 75\% экрана + дочерний UIView зеленого цвета, отступ 100 от правой границы родителя, ширина и высота 50, второй дочерний UIView коричневого цвета, отступ -100 от левой границы родителя, ширина и высота 50, третий дочерний UIView зеленого цвета, отступ 10 от верхний границы родителя, ширина и высота 50 & Тест пройден успешно  \\ \hline
   21 & UIView красного цвета, 75\% экрана + дочерний UIView зеленого цвета, отступ 100 от правой границы родителя, ширина и высота 50, второй дочерний UIView коричневого цвета, отступ -100 от левой границы родителя, ширина и высота 50, третий дочерний UIView зеленого цвета, отступ 10 от верхний границы родителя, ширина и высота 50 + дочерний UILabel, цвет фона --- красный, текст --- <<Привет>>, выравнивание текста по правому краю, отступ 10 от верхней границы родителя & Тест пройден успешно  \\ \hline  \end{tabular}
 \end{center}
\end{table}


\pagebreak