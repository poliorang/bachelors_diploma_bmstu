\section*{ВВЕДЕНИЕ}
\addcontentsline{toc}{section}{ВВЕДЕНИЕ}

В настоящее время, с развитием технологий и повсеместным использованием интернета, наличие интерфейса играет ключевую роль в обеспечении комфортного и эффективного взаимодействия пользователей с различными приложениями и веб-сервисами, а профессии мобильного или frontend разработчика являются одними из самых востребованных на рынке. 

Вёрстка для мобильных и frontend разработчиков является сложным и трудоемким процессом из--за необходимости учитывать различные размеры экранов, плотности пикселей, ориентации устройств и другие аспекты, которые влияют на отображение контента на различных устройствах. 
Это требует создания адаптивного дизайна, который будет корректно отображаться как на больших, так и на маленьких устройствах. 
Также стоит учитывать, что изменения в дизайне или структуре приложения могут потребовать переработки большого объема кода, что может быть затратным по времени и ресурсам. 

С целью разрешить описанную ранее проблему было создано большое количество методов создания пользовательского интерфейса. 
Однако не каждый из них предоставляет возможность мгновенно отображать изменения при внесении коррективов в код. 
Целью данной работы является разработка метода динамического отображения изменений пользовательского интерфейса на основе обработки изменений XML--файла.

Для достижения поставленной цели необходимо решить следующие задачи:

\begin{itemize}[label=---]
	\item выявить критерии, по которым можно классифицировать методы отображения пользовательского интерфейса;
	\item рассмотреть существующие методы отображения интерфейса и классифицировать их согласно выдвинутым критериям;
	\item разработать метод динамического отображения изменений интерфейса на основе обработки изменений XML--файла;
	\item разработать программное обеспечение, реализующее представленный метод;
	\item сравнить скорость внесения изменений в интерфейс разработанной и существующей реализаций.
\end{itemize}

\pagebreak
